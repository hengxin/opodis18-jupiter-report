% file: parts/wlspec.tex

%%%%%%%%%%%%%%%%%%%%
\begin{frame}{}
  \centerline{\Huge \teal{Weak List Specification}}
\end{frame}
%%%%%%%%%%%%%%%%%%%%

%%%%%%%%%%%%%%%%%%%%
\begin{frame}{}
  \fig{width = 0.80\textwidth, frame}{figs/podc16-attiya}

  \vspace{0.20cm}
  \begin{definition}[Weak List Specification \wlspec{}~\ncite{Attiya:PODC16}]
    Informally, \wlspec{} requires the ordering between \red{elements that are not deleted} to be consistent across the system.
  \end{definition}

  \vspace{0.60cm}
  \centerline{\red{\large Specify a global property \teal{\it on all states} across the system.}}
\end{frame}
%%%%%%%%%%%%%%%%%%%%

%%%%%%%%%%%%%%%%%%%%
\begin{frame}{}
  \begin{center}
    We show that \wlspec{} can be rephrased as
  \end{center}

  \begin{definition}[Pairwise State Compatibility Property]
    For any \red{pair} of list states, there \red{cannot} be two elements $a$ and $b$ \\
    such that \purple{$a$ precedes $b$ in one state} but \purple{$b$ precedes $a$ in the other}.
  \end{definition}

  \vspace{0.30cm}
  \begin{columns}
    \column{0.50\textwidth}
      \fig{width = 0.45\textwidth}{figs/ex-weak-list-spec}
      \vspace{-0.60cm}
      \fig{width = 0.30\textwidth}{figs/red-cross}
    \pause
    \column{0.50\textwidth}
      \fig{width = 0.55\textwidth}{figs/ex-strong-list-spec}
      \vspace{-0.60cm}
      \fig{width = 0.20\textwidth}{figs/green-check}

      \vspace{-1.00cm}
      \begin{center}
	\violet{\footnotesize Prohibited by ``Strong List Specification''}
      \end{center}
  \end{columns}
\end{frame}
%%%%%%%%%%%%%%%%%%%%
