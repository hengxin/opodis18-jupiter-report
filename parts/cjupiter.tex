% file: parts/cjupter.tex

%%%%%%%%%%%%%%%%%%%%
\begin{frame}{}
  \centerline{\Huge \teal{CJupiter (Compact Jupiter)}}
\end{frame}
%%%%%%%%%%%%%%%%%%%%

%%%%%%%%%%%%%%%%%%%%
\begin{frame}{}
  \begin{center}
    {\large CJupiter maintains an \hl{$n$-ary ordered state space} for each replica.}
  \end{center}

  \fig{width = 0.40\textwidth}{figs/cjupiter-allinone}

  \begin{center} 
    There can be \red{more than two edges} coming from the same node. \\[5pt] \pause
    Edges from the same node are \red{totally ordered} according to \\[2pt] 
    the \red{serialization order} of associated operations.
  \end{center}
\end{frame}
%%%%%%%%%%%%%%%%%%%%

%%%%%%%%%%%%%%%%%%%%
\begin{frame}{}
  \begin{Theorem}[Equivalence of CJupiter and Jupiter]
    Under the same schedule, the behaviors of corresponding replicas in CJupiter and Jupiter are the same.
  \end{Theorem}

  \vspace{0.80cm}
  \begin{columns}
    \column{0.80\textwidth}
      \begin{description}
	\item[Schedule:] \textsc{Issue}, \textsc{Send}, and \textsc{Receive} of operations
	\item[Behavior:] A sequence of replica states
      \end{description}
  \end{columns}

  \pause
  \vspace{0.50cm}
  \begin{center}
    {\large \hl{Equivalence} from the perspectives of both the server and clients.}
  \end{center}
\end{frame}
%%%%%%%%%%%%%%%%%%%%

%%%%%%%%%%%%%%%%%%%%
\begin{frame}{}
  \centerline{\teal{\large At the server side:}}
  \begin{prop}[$n \leftrightarrow 1$ (Informal)]
    The \cyan{single $n$-ary ordered state space} at the server side in CJupiter \\
    is a \red{union} of \cyan{$n$ $2D$ state spaces} at the server side in Jupiter.
  \end{prop}

  \pause
  \vspace{1.20cm}
  \centerline{\teal{\large At the client side:}}
  \begin{prop}[$1 \leftrightarrow 1$ (Informal)]
    Jupiter is \red{slightly optimized in implementation} at clients by eliminating redundant OTs in CJupiter.
  \end{prop}
\end{frame}
%%%%%%%%%%%%%%%%%%%%

%%%%%%%%%%%%%%%%%%%%
\begin{frame}{}
  \begin{prop}[Compactness of CJupiter (Informal)]
    {\large At a high level, CJupiter maintains only \red{one} $n$-ary ordered state space.}
  \end{prop}

  \begin{center}
    \resizebox{0.40\textwidth}{!}{\input{tikz/cjupiter-allinone-path}}
    % \fig{width = 0.40\textwidth}{figs/cjupiter-allinone-path}

    \pause
    All replica states are represented in a single data structure. \\[5pt] \pause
    Each replica behavior corresponds to a \red{path} going through this state space.
  \end{center}
\end{frame}
%%%%%%%%%%%%%%%%%%%%
